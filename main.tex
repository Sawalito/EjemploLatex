\documentclass[a4paper,12pt]{article}
\usepackage[utf8]{inputenc}
\usepackage{graphicx}
\usepackage{amsmath}
\usepackage{amssymb}
\usepackage{hyperref}
\usepackage{geometry}
\usepackage{fancyhdr}
\usepackage{booktabs}

\geometry{margin=1in}
\pagestyle{fancy}
\fancyhf{}
\rhead{Saúl — Ciencia de Datos, ITAM}
\lhead{Proyecto con Datos}
\cfoot{\thepage}

\title{Análisis Exploratorio de Datos}
\author{Saúl \\ Ciencia de Datos — ITAM}
\date{\today}

\begin{document}

\maketitle

\section*{Resumen}

Este informe presenta un análisis exploratorio de datos usando herramientas estadísticas básicas y visualizaciones.  

\section{Introducción}

Este proyecto tiene como objetivo explorar un conjunto de datos y aplicar técnicas estadísticas simples para obtener conclusiones preliminares. Los datos provienen de fuentes abiertas y se encuentran disponibles en este repositorio.

\section{Formulación del Problema}

El objetivo es estimar la media y varianza de una variable aleatoria \( X \), asumiendo que sigue una distribución normal:

\[
X \sim \mathcal{N}(\mu, \sigma^2)
\]

La media muestral se calcula como:

\[
\bar{X} = \frac{1}{n} \sum_{i=1}^n X_i
\]

Y la varianza muestral es:

\[
S^2 = \frac{1}{n - 1} \sum_{i=1}^n (X_i - \bar{X})^2
\]

\section{Tabla Resumen}

\begin{table}[h!]
\centering
\begin{tabular}{@{}lcc@{}}
\toprule
Variable & Media (\(\bar{X}\)) & Desviación estándar (\(S\)) \\
\midrule
Altura (cm) & 170.3 & 6.2 \\
Peso (kg)   & 68.5  & 7.4 \\
\bottomrule
\end{tabular}
\caption{Estadísticas descriptivas}
\end{table}

\section{Visualización}

\section{Conclusión}

Los resultados indican que la distribución de la variable es aproximadamente normal. En el futuro se pueden aplicar pruebas de hipótesis y modelos de regresión.

\end{document}
